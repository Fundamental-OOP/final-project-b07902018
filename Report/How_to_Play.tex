\section{How to Play}

玩家進入遊戲後,可以選擇遊玩人數及遊戲世界,目前提供一至二人進行遊戲,並有四個遊戲地圖供玩家選擇,玩家透過點擊視窗上的遊玩人數及遊戲世界的數字進行可做選擇,選擇完畢後點擊START按鈕即可進入遊戲。

在進入遊戲後,可見遊戲本體在視窗中央,視窗右方從上到下分別顯示遊戲倒數、玩家目前分數及食譜,而視窗下方則顯示目前的訂單,透過達成訂單要求即可獲得分數。
 
玩家一可利用鍵盤按鍵W, S, A及D分別進行上、下、左及右的移動,並可利用鍵盤按鍵Q觸發食材的提取, 並以鍵盤按鍵E觸發食材的放置; 玩家二可利用鍵盤按鍵I, K, J及L分別進行上、下、左及右的移動,並可利用鍵盤按鍵U觸發食材的提取, 並以鍵盤按鍵Q觸發食材的放置。

遊戲中,可進行食材提取的物件共有7個, 分別是EggBasket, BreadBasket, CheeseBlock, SpinachGarden, PieBox, FruitBasket及TomatoBasket, 除了FruitBasket會隨機給予Apple, Banana及Orange其中一者之外, 其他的物件都只會進行單一一種食材的生成。

若有4種類型物件可進食材的放置, 分別是可臨時放置至多一項食材的WoodPlatform, 可進行食材棄置的TrashCan, 用於製作產品的ApplePieStove, SaladBowl, SandwichMaker與FriedEggStove及提交最終產品的PickupWindow。

玩家可參考右方的食譜進行產品的製作,並將訂單相對應的產品放置於PickupWindow,若所放置的食材滿足其中一項訂單,則可獲得相對應的成績,並可見ScoreBoard即時更新成績,倘若所放置的食材不滿足任何一項訂單,則會受到分數懲罰,每次將倒扣10分直到分數為0。

訂單會隨時間增加,至多累績至5筆訂單。

當時間倒數至0時, 遊戲會強制中止並顯示玩家的分數, 玩家可透過點擊Play Again回到Menu再次進行遊戲。

